\documentclass[12pt,a4paper]{article}
\usepackage{amsmath}
\begin{document}
	
% Folha de rosto
\begin{titlepage}
	\centering
	\vspace{2cm}
	
	{UNIVERSIDADE FEDERAL FLUMINENSE} \\ [0.1cm]
	{BACHARELADO EM CIÊNCIA DA COMPUTAÇÃO} \\ [0.1cm]
	{TCC00288 - BANCO DE DADOS II}
	
	\vfill
	
	{\Large \bfseries Avaliação Continuada 1: Estudo de Índices}
	
	\vfill
	
	{BEATRIZ DE OLIVEIRA PIEDADE}
	
	\vfill
	{NITERÓI} \\
	{2024}
\end{titlepage}

\tableofcontents

% Conteúdo
\newpage
\section{Introdução}

Para realizar consultas mais rápidas e eficientes nos bancos de dados utilizamos estruturas chamadas de índices. É possível criar um índice adaptado ao tipo da consulta a ser realizada. Alguns dos tipos disponibilizados pelo PostgreSQL são:

\begin{description}
	\item[Índice Árvore-B:] Tipo padrão do PostgreSQL. Utilizado para otimizar a busca em colunas com valores repetidos.
	\item[Índice Hash:] Utilizado para consultas de igualdade em colunas, preferencialmente aquelas com valores uniformemente distribuídos.
	\item[Índice GIN:] Utilizado geralmente para pesquisa de texto e listas.
	\item[Índice GiST:] Tipo genérico de índice que funciona como uma estrutura para a criação de índices personalizados. Estes índices personalizados são muito utilizados na visão computacional e na bioinformática.
\end{description}

\noindent
Para criar um índice no PostgreSQL, é necessário utilizar a seguinte sintaxe:

\begin{verbatim}
	CREATE INDEX nomeDoIndice
	ON nomeDaTabela
	USING tipoDoIndice (coluna0, ..., colunaN);
\end{verbatim}

\newpage
\section{Esquema}

\begin{flushleft}
	Cliente (\underline{id}, nome, endereco, telefone) \\[0.2cm]
	EmpresaColaboradora (\underline{id}, nome, cnpj, endereco, telefone) \\[0.2cm]
	Pf (\underline{id}, cpf, salario) \\
	\quad \quad \quad \quad id REFERENCIA Cliente(id) \\[0.2cm]
	Pj (\underline{id}, cnpj, lucro) \\
	\quad \quad \quad \quad id REFERENCIA Cliente(id) \\[0.2cm]
	Compra (\underline{id}, idCliente, data, hora, nfe) \\
	\quad \quad \quad \quad idCliente REFERENCIA Pf(id) \\[0.2cm]
	Entrega (\underline{id}, data, idEntregador, idCompra) \\
	\quad \quad \quad \quad idEntregador REFERENCIA EmpresaColaboradora(id) \\
	\quad \quad \quad \quad idCompra REFERENCIA Compra(id) \\[0.2cm]
	Produto (\underline{id}, nome, descricao, precoUnitario, idFornecedor) \\
	\quad \quad \quad \quad idFornecedor REFERENCIA EmpresaColaboradora(id) \\[0.2cm]
	ItemCompra (\underline{id}, idProduto, idCompra, quantidade) \\
	\quad \quad \quad \quad idProduto REFERENCIA Produto(id) \\
	\quad \quad \quad \quad idCompra REFERENCIA Compra(id) \\[0.2cm]
	Devolucao (\underline{id}, data, idItemCompra) \\
	\quad \quad \quad \quad idItemCompra REFERENCIA ItemCompra(id) \\
\end{flushleft}

\newpage
\section{Análise das Consultas}

\subsection{Clientes com nome contendo "João"}

\subsection{Compras feitas após o dia 1º de janeiro de 2023}
\subsection{Produtos com preço unitário acima de 100,00}
\subsection{Devoluções realizadas antes do dia 1º de junho de 2023}
\subsection{Empresas colaboradoras localizadas em um endereço contendo "Centro"}
\subsection{Clientes cujo telefone começa com "21"}
\subsection{Itens de compra que possuem uma quantidade maior que 5}
\subsection{Entregadores (empresas colaboradoras) que participaram de uma entrega no ano de 2024}
\subsection{Compras feitas por clientes com salário superior a 5.000,00}
\subsection{Produtos fornecidos por empresas cujo nome contém "Tech"}
\subsection{Clientes que não possuem endereço registrado (endereço nulo)}
\subsection{Devoluções feitas para itens de compras de produtos cujo preço unitário é maior que 200,00}
\subsection{Clientes que fizeram pelo menos uma compra em fevereiro de 2023}
\subsection{Entregas que ocorreram após as 01/01/2001}
\subsection{Registros de Pessoas Jurídicas com lucro superior a 1.000,00}

\newpage
\section{Conclusão}
\end{document}